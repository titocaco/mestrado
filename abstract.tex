% !TEX root = main.tex
\chapter*{Abstract}
\addcontentsline{toc}{chapter}{Abstract}

In a whole universe of multimedia applications, the observation of information tends to be a composition typically resulting from some mixing process. For some applications, such \textit{Status quo} may be exactly what one wishes, however, as is not always the case, it is desirable for each generating source to be represented in a different information channel; the task carrying out such a process is known as a \textit{source separation} or \textit{signal separation}, regardless of what these signals represent. Such a task is not limited to a single method; there are many different ways to carry out the separation process.

In this work, GAN (\textit{Generative Adversarial Networks}) frameworks were explored, coupled with more traditional techniques to perform the task of source separation aiming to provoking an audio enhancement effect for human voice signals, which includes a noise attenuation process.

\vspace{2.0cm}

\noindent \textbf{Palavras-chave:} \hspace{1.0cm} \textit{Generative Adversarial Networks}, source separation, signals separation, voice enhancement, audio signals.