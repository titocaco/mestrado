% !TEX root = main.tex
\chapter*{Resumo}
\addcontentsline{toc}{chapter}{Resumo}

Em todo um universo de aplicações multimídia, a observação de informações tende a ser uma composição tipicamente resultante de algum processo de mistura. Para algumas aplicações, tal \textit{Status quo} pode ser justamente o que se almeja, contudo, como nem sempre é o caso, é desejável que cada fonte geradora seja representada em um canal diferente de informação; a tarefa que realiza tal processo é conhecida como \textit{separação de fontes} (ou \textit{separação de sinais}), independentemente do que representam esses sinais. Tal tarefa não é limitada a um único método; existem muitas formas distintas para se realizar o processo de separação.

Neste trabalho, foram explorados arcabouços de GANs (\textit{Generative Adversarial Networks}) aliados a técnicas mais tradicionais para se realizar a tarefa de separação de fontes com o objetivo de se provocar um efeito de aprimoramento de áudio para sinais de vozes humanas, o que inclui um processo de atenuação de ruído.

\vspace{2.0cm}

\noindent \textbf{Palavras-chave:} \hspace{1.0cm} \textit{Generative Adversarial Networks}, separação de fontes, separação de sinais, aprimoramento de voz, sinais de áudio.